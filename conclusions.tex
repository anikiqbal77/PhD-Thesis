\chapter{Conclusions}
\label{sec:Conclusions}

To conclude, we studied the effect of electron interaction on the collective and p-h excitations in a laterally confined two-dimensional Fermi gas.
The spectrum of confined electrons splits into subbands of transverse quantization.
We focused on the transversal excitations across the channel controlled by the subband separation energy scale.

When the Zeeman splitting is tuned to the subband separation a special type of  resonance, the BSR is detected in the dc transport measurements of Ref.~\cite{Frolov2009,Frolov2009a}.
Hence we analyzed the class of collective modes adiabatically connected to the spin flip intersubband p-h excitations which become soft at the BSR. 
The BSR measurements of \cite{Frolov2009} probe the individual p-h excitations.
We showed however relying on the perturbation theory that for a weak, short range interaction and sufficiently close to the BSR such p-h excitations are indistinguishable from the collective spin sloshing mode.
Therefore finding the frequency of the collective spin sloshing mode at the same time allows us to determine the shift of the BSR caused by interactions.

To meet this goal we applied the Fermi liquid theory to identify and analyze the spin sloshing mode.
It combines the features of two other more familiar collective excitations.
The first is the density oscillations across the channel, with amplitude vanishing at the center of the channel and is referred to as sloshing, or Kohn mode. 
The second is the collective spin precession mode in the presence of a Zeeman splitting.

The spin-sloshing mode is the collective spin precession with an amplitude growing linearly away from the center of the channel, see Fig.~\ref{fig:slosh}.
We demonstrated that the Kohn theorem does not apply to it, and found its frequency renormalization due to a weak and short range interaction.

Finally, by combining the above results we obtained the physical picture of interaction effect on the BSR.
We start by setting the Zeeman splitting to the subband separation in a Fermi gas.
Then the spin-flip intersubband p-h excitations have a zero energy, and the system is at resonance.
When a weak interaction is turned on these p-h excitations remain degenerate with their common energy becoming non-zero.
Hence, the interaction detunes the resonance.
To tune it back the Zeeman splitting must be accordingly adjusted.
It follows that the resonant magnetic field is modified by interactions. 
To quantify this statement we used the analytical expression for the energy of a spin sloshing mode was obtained within the Fermi liquid theory.

In our analysis we assumed the confining potential to be parabolic with the equidistantly separated subbands, see Eq.~\eqref{M17}.
Although the realistic confining potential is never strictly harmonic, the deviation from parabolic potential profile is in many cases small. 
If we take the non-parabolic part of the confining potential in the form,
$\delta V_{np}(z) = (\epsilon/4)z^4$ then the continuum of the p-h excitations is split due to the variation of the level shift $\delta E_{n\alpha}(k_x) \approx \langle n | \delta V_{np}| n \rangle $ with the subband index $n$.
The unharmonicity effect is negligible provided the energy shift due to the interactions, $\omega_c (V_{\rho} \! -\!  3 V_s)\nu$, Eq.~\eqref{result1_b} exceeds the typical value of $\delta E_{n\alpha}(k_x)$. 
This gives the upper bound, $\epsilon < (m \omega_c^2/ 2 E_F)^2 \omega_c m (V_{\rho} \! -\!  3 V_s)$, as we estimate $\langle z^4 \rangle \sim  (2 E_F/ m \omega_c^2)^2$.

In summary, the interaction induced shift of the BSR is obtained by tracing the frequency renormalization of the collective spin sloshing mode.
The shift of the BSR is a consequence of the non-existence of the Kohn theorem for this mode.

The hydrodynamic description requires short equilibration length $\ell$. Thus validity of our theory is limited by the condition $\ell\ll a$, which imposes certain restriction on temperature. Specifically, for the Fermi liquids $\ell=v_F\tau_{ee}$ is determined by collisions with the typical rate $\tau^{-1}_{ee}\sim T^2/E_F$. Since $\omega_\perp\sim v_F/a$, hydrodynamic regime is realized at temperatures $T>T_h$ above the crossover scale $T_h\sim\sqrt{\omega_\perp E_F}\sim E_F/\sqrt{N}$, where $N$ is the number of occupied sub-bands of the transversal quantization. It also follows that with necessity hydrodynamics requires $T\gg\omega_\perp$. While this inequality is reasonably satisfied for the cold gases that are confined by a very shallow potential, it obviously breaks in the ultra-cold limit where collisionless regime prevails. In the latter case attenuation coefficient of the Kohn mode is expected to follow quadratic temperature dependence $\tau^{-1}_{1}\propto \alpha T^2/E_F$ based on the Pauli principle and phase space restrictions argument, whereas in the hydrodynamic regime $\tau^{-1}_{1}\propto 1/T^2$ in accordance with Eq.~(\ref{tau-K}). The nonmonotonic temperature dependence of the decay rate has been observed experimentally~\cite{Riedl}.  

Our hydrodynamic approach has interesting parallels with the Luttinger liquid description of collective modes in confined inhomogeneous one-dimensional gases ~\cite{Citro}. The eigenvalue equation for the normal eigenmodes in that case, analogous to our Eq.~(\ref{normal-modes-eq}), is given by  
\begin{equation}\label{eigenmodes-LL}
-\omega^2_n\chi_n(z)=v(z)K(z)\partial_z\left(\frac{v(z)}{K(z)}\partial_z\chi_n(z)\right)
\end{equation}
where Luttinger liquid interaction parameter satisfies the relation\\ $v(z)K(z)=\pi \rho(z)/m^2$. This equation is supplemented by the boundary condition $\chi_n(\pm a)=0$ and normalization condition $\int^{a}_{-a} dz \chi_j(z)\chi_j(z)/v(z)K(z)=\delta_{ij}$. For the particular choice of $v(z)=v_0\sqrt{1-z^2/a^2}$ and $K(z)=K_0(1-x^2/a^2)^\gamma$ the solutions $\chi_n(z)$ are obtained in terms of Gegenbauer polynomials with the spectrum of excitations $\omega^2_n=(v_0/a)^2(n+1)(n+2\gamma+1)$~\cite{Citro,Petrov1,Stringari}. In the model of $\gamma=2$, the problem simplifies to the case of Legendre polynomials~\cite{Petrov2} with the spectrum of excitations analogous to our result (\ref{omega-n}). Another interesting limit is $\gamma=0$, which corresponds to the case of the Tonks-Girardeau gas, where the Gegenbauer polynomials reduce to Chebyshev polynomials. Inclusion of dissipative terms into Eq.~(\ref{eigenmodes-LL}) requires consideration of corrections to Luttinger liquid model which account for the inelastic scattering of bosons and ultimately describe equilibration processes. As recently shown such generalization is possible both in the limit of weak \cite{AL} and strong \cite{Matveev} interactions and application of this formalism to the problem of decay of collective modes is an interesting problem for future research. Along this rout one may hope to find a unified description, which interpolates between the quantum~\cite{Matveev} and classical~\cite{Andreev} hydrodynamic regimes of Luttinger liquids, and which is broadly applicable for arbitrarily strong interactions.  

In chapter 4, we established that for a confined Bose-Einstein condensate coupled with its thermal cloud in a pancake geometry, the collective modes are not affected by interactions and the Kohn theorem holds. However, for any departure of the confinement potential from parabolicity, the collective modes are shifted in frequency.

